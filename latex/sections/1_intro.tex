\chapter{Introduction}\label{ch:Intro}

The exploration of Mars has been significantly advanced with the success of the NASA Perseverance rover mission, whose primary objective is to search for signs of past microbial life and collect samples of Martian rock and soil. A key component of this mission is the \textit{Ingenuity} helicopter (Figure \ref{fig:ingenuity}), which demonstrates the potential of aerial vehicles for planetary exploration. Ingenuity's flights provide valuable data on the Martian atmosphere and, in general, environmental conditions that can be leveraged for future missions.
\begin{figure}[htbp]
    \centering
    \includegraphics[width=0.3\textwidth]{images/ingenuity}
    \caption{Ingenuity at Wright Brothers Field on 6 April 2021, its third day of deployment on Mars. Image source: Wikipedia.}
    \label{fig:ingenuity}
\end{figure}

\section{The Ingenuity helicopter}
Unlike the more common quadrotors, Ingenuity is a \textit{co-axial} helicopter with two counter-rotating rotors stacked on a single mast. This design choice allows for a more compact structure that does not need a tail rotor to counteract torque. Instead of controlling motion through varying the speed of the two rotors, Ingenuity uses a mechanism called \textit{swashplate} \cite{Grip2019} (see Figure \ref{fig:swashplate}) to change the pitch of the rotor blades cyclically and collectively.
\begin{itemize}
    \item \textit{Cyclic pitch} changes the angle of the blades as they rotate, which tilts the thrust vector allowing to generate the forces and moments needed for horizontal motion (roll and pitch).
    \item \textit{Collective pitch} adjusts the angle of all blades of a rotor simultaneously, controlling the total magnitude of the thrust for vertical motion.
\end{itemize}
This actuation method yields flight dynamics that are fundamentally different from those of multi-rotor drones.
\begin{figure}[htbp]
    \centering
    \includegraphics[width=0.6\textwidth]{images/swashplate.png}
    \caption{A helicopter swashplate mechanism. It translates non-rotating control inputs from the servos into the rotating frame of the rotor blades, controlling both collective and cyclic pitch. Image source: COMSOL Application Gallery.}
    \label{fig:swashplate}
\end{figure}

\subsection{Control}
Controlling Ingenuity is a challenging task due to two main reasons. First, the Martian atmosphere is much thinner than Earth's, with a density of about 1\% that of sea level on Earth. This results in reduced aerodynamic forces and moments, making it difficult to generate sufficient lift and, in general, it alters the helicopter's flight dynamics. This requires the two rotors to spin at very high speeds (over 2500 RPM) to generate enough lift for takeoff and maneuvering, leading to complex aerodynamic effects \cite{Balaram2018}.

Second, the helicopter's dynamics are nonlinear and coupled. In fact, the forces generated by the rotors depend on the vehicle's orintation, and the translational and rotational motions are linked. While linear controllers have been succesfully used for Ingenuity's flight control \cite{Grip2019}, their performance can be limited, especially during aggressive maneuvers or in the presence of disturbances such as wind gusts. In general, standard linear control techniques tend to struggle in handling these nonlinearities.

\subsubsection{Dynamic feedback linearization}
We adopt an approach from nonlinear control theory known as \textit{dynamic feedback linearization} (DFL) \cite{vendittelli:quad_dfl} that aims to cancel out the coupled nonlinearities in the system dynamics through feedback, resulting in a decoupled linearized system that can be controlled using linear control methods. The main idea is to find a suitable change of coordinates and a nonlinear control law that cancels out the unwanted nonlinear terms in the dynamics.

The "dynamic" aspect of DFL refers to an extension of the method that is necessary when the control inputs do not immediately affect the outputs, requiring the controller to have its own internal dynamics. As we will show, this is precisely the case for controlling the position of a co-axial helicopter like Ingenuity using the swashplate mechanism.

\section{Objective}
The main goal of this project is to formally analyze the dynamic feedback linearizability of the Ingenuity helicopter, design a state feedback controller for \textit{trajectory tracking}, and validate its performance and robustness through turbances and simulations. Specifically, we aim to:
\begin{enumerate}
    \item Develop a dynamic model of Ingenuity based on Newton-Euler equations, capturing the key dynamics and couplings introduced by the co-axial rotor configuration;
    \item analyze such dynamics to determine the conditions under which the system can be linearized using dynamic feedback;
    \item design the nonlinear control law that achieves linearization and decoupling;
    \item implement a tracking controller for the linearized system;
    \item evaluate the performance of our controller in simulation under various scenarios and for different maneuvers.
\end{enumerate}

\section{Structure}
The report is organized as follows. In Chapter \ref{ch:Dynamics} we present the detailed dynamic model of the Ingenuity helicopter deriving the equations of motion. In Chapter \ref{ch:Control} we provide a mathematical formulation of the dynamic feedback linearization technique and derive the specific control law for our system. In Chapter \ref{ch:Setup} we describe the simulation environment, the implementation details of the controller, and the test scenarios. In Chapter \ref{ch:Results} we present the results of our experiments and evaluate the controller's performance. Finally, in Chapter \ref{ch:Conclusion} we summarize our findings and discuss potential future work.