\chapter{Simulation setup}\label{ch:Setup}

In this chapter, we describe the simulation environment used to validate the control law computed in Chapter \ref{ch:Control}. We define the physical parameters of the Ingenuity helicopter, the controller tuning, the reference trajectories used for testing, the external disturbances applied to check robustness, and the performance metrics used to evaluate the results.

\section{Simulation environment}
The simulation is implemented in MATLAB. Since the system dynamics derived in Chapter \ref{ch:Dynamics} are nonlinear and continuous in time, we use a standard numerical solver for ordinary differential equations (\texttt{ode15s}) to integrate the system dynamics.

All the simulations are run with a variable time step to ensure numerical stability, but the results are analyzed and plotted over a fixed time span of $T=30$ seconds. The initial condition for all simulations corresponds to Ingenuity being on the ground (zero altitude) and the rotors spinning to generate a thrust equal to the vehicle's weight (hovering condition), to simulate a realistic takeoff scenario.

\section{Model parameters}
The physical parameters used in the simulation represent the Ingenuity helicopter operating in the Martian environment. The mass and gravity acceleration are taken from the mission data \cite{Grip2019}. The inertia tensor is approximated as a diagonal matrix, and the aerodynamic and drag coefficients are estimated to provide a realistic damping effect. Table \ref{tab:parameters} summarizes these parameters.

\begin{table}[htbp]
    \centering
    \caption{Physical parameters of the Ingenuity model.}
    \label{tab:parameters}
    \begin{tabular}{lcc}
        \hline
        \textbf{Parameter} & \textbf{Symbol} & \textbf{Value} \\
        \hline
        Mass & $m$ & $1.8$ kg \\
        Mars Gravity & $g$ & $3.69$ m/s$^2$ \\
        Inertia ($x$-axis) & $I_{xx}$ & $0.02$ kg$\cdot$m$^2$ \\
        Inertia ($y$-axis) & $I_{yy}$ & $0.02$ kg$\cdot$m$^2$ \\
        Inertia ($z$-axis) & $I_{zz}$ & $0.03$ kg$\cdot$m$^2$ \\
        Translational drag & $\bs{A}_\text{trans}$ & $\text{diag}(0.05, 0.05, 0.1)$ N$\cdot$s/m \\
        Rotational drag & $\bs{A}_\text{rot}$ & $\text{diag}(0.01, 0.01, 0.05)$ N$\cdot$m$\cdot$s \\
        \hline
    \end{tabular}
\end{table}

\section{Controller implementation}
\subsection{Gain tuning}
The dynamic feedback linearization controller transforms the nonlinear system into decoupled chains of linear integrators. We stabilize these chains by assigning the poles of the error dynamics.

For the position subsystem, which has a relative degree of $r=4$, the error dynamics are described by the linear differential equation:
\[ e^{(4)} + k_3 e^{(3)} + k_2 \ddot{e} + k_1 \dot{e} + k_0 e = 0 \]
To ensure a fast response with no overshoot, we placed all four poles at $\lambda = -2$ on the real axis. By expanding the characteristic polynomial $(s+2)^4$, we obtained the following gains:
\[ \bs{k}_p = [k_0, k_1, k_2, k_3] = [16, 32, 24, 8] \]
These scalars define the diagonal gain matrices used in Eq. \eqref{eq:position_controller} as $\bs{K}_i = k_i \bs{I}_3$, ensuring identical behavior for the $x, y, z$ axes.

For the yaw subsystem ($r=2$), we selected proportional and derivative gains to ensure stable tracking:
\[ \bs{k}_\psi = [k_{p,\psi}, k_{d,\psi}] = [4, 4] \]

\subsection{Actuator saturation}
Real actuators cannot produce infinite forces and torques. To make the simulation realistic, we implemented saturation limits on the control inputs based on the known capabilities of Ingenuity's rotors \cite{Grip2019}.

\paragraph{Thrust limits}
The thrust produced by the rotors is limited by the rotational speed and the atmospheric density. The thrust-to-weight ratio (TWR) of Ingenuity is between 1.3 and 1.6; we selected a nominal maximum limit of 145\%:
\[ F_\text{max} = 1.45 \cdot mg \approx 9.63 \text { N} \]
We also enforced a minimum thrust limit of $F_\text{min}=0.3 mg$ to prevent the total thrust from reaching zero, which would cause a singularity in the decoupling matrix $\bs{J}$ (see Chapter \ref{ch:Control}).

\paragraph{Torque limits}
The control torques $\tau_\phi, \tau_\theta, \tau_\psi$ are saturated at $\pm 0.05$ Nm to represent the physical limits of the cyclic and collective pitch mechanism.

\section{Reference trajectories}
We define a set of reference trajectories to test the performance of the controller under different conditions. They are chosen so as to evaluate the controller's ability to track both simple and smooth paths, as well as to more aggressive maneuvers.

The controller requires the reference to be smooth up to the fourth derivative (snap) for the position; therefore, all trajectories are generated analytically or via polynomials.

\paragraph{Box trajectory} This trajectory simulates a patrol mission: the helicopter takes off to an altitude of 5 meters and then follows a square path with a side length of 10 meters. The motion along each segment is generated using quintic polynomials to ensure smooth transitions in position, velocity, acceleration, jerk, and snap. The yaw angle is kept constant at zero throughout the trajectory.

\paragraph{Helix trajectory} This trajectory requires the helicopter to ascend while performing a circular motion. It is defined analyitically as:
\[  \begin{cases}
x_d(t) = R \cos(\omega_\text{ref}t) \\
y_d(t) = R \sin(\omega_\text{ref}t) \\
z_d(t) = \min(v_z t, z_\text{max}) \\
\end{cases} \]
where the radius is $R=2$ m, the angular rate is $\omega_\text{ref} = 0.5$ rad/s, and the vertical velocity is $v_z = 0.2$ m/s.

\paragraph{Figure-8 trajectory} This trajectory forces the helicopter to track a figure-8 path at a constant altitude of 5 meters. The planar equations are:
\begin{equation}
    \begin{cases}
        x_d(t) = A \cos(\omega_\text{ref} t) \\
        y_d(t) = \frac{A}{2} \sin(2\omega_\text{ref} t) \\
        z_d(t) = z_\text{ref}
    \end{cases}
\end{equation}
where the scale is $A=3.0$ m and the frequency is $\omega_\text{ref} = 0.4$ rad/s.

\section{Disturbances}\label{sec:disturbances}
To verify the robustness of the controller, we simulate an external wind gust disturbance. The wind is modeled as a constant force vector $\bs{F}_\text{wind}$ acting on the center of mass in the inertial frame.

The disturbance can be configured in terms of:
\begin{itemize}
    \item \textit{Direction}: the wind can act along any combination of the $x,y,z$ inertial axes.
    \item \textit{Intensity}: a randomized magnitude within a specified range (e.g., $\pm 2$ N) is applied to the active axes.
    \item \textit{Timing}: The wind is active only during a specific time window $[t_\text{start}, t_\text{end}]$, typically set to $t \in [10, 20]$ s to observe the response of the controller.
\end{itemize}

In the equations of motion, this force is rotated into the body frame and added to the total force:
\begin{equation}
    \bs{F}_\text{tot}^b = \bs{F}_\text{thrust}^b + \bs{F}_\text{gravity}^b + \bs{F}_\text{drag}^b + \bs{R}^T \bs{F}_\text{wind}
\end{equation}
Since the DFL controller relies on model cancellation, this external force acts as an unmodeled disturbance that the outer linear feedback loop must reject.

\section{Performance metrics}
To quantitatively evaluate the controller, we compute the following metrics based on the discrete simulation data sampled at time steps $t_k$, for $k=1, \dots, N$.

\paragraph{Robustness to disturbances} To assess the controller's ability to reject unmodeled dynamics, we compute the maximum tracking error specifically during the wind gust interval ($t \in [t_\text{start}, t_\text{end}]$ s):
\begin{equation}
    e_\text{gust} = \max_{t_k \in [t_\text{start}, t_\text{end}]} \|\bs{e}_P(t_k)\|
\end{equation}

\paragraph{Actuator saturation} We evaluate the \textit{feasibility} of the control effort by calculating the percentage of flight time where the thrust input reached its physical saturation limits ($F_\text{min}$ or $F_\text{max}$):
\begin{equation}
    S_{\%} = \frac{N_\text{sat}}{N} \cdot 100
\end{equation}
A non-zero value indicates that the controller requested more power than the actuators could provide, leading to unavoidable tracking degradation.