\chapter{Control design}\label{ch:Control}

This chapter details the design of a nonlinear controller for the Ingenuity helicopter model developed in Chapter \ref{ch:Dynamics}. Due to the underactuatued and coupled nature of its dynamics, standard linear control techniques are not suitable for achieving trajectory tracking. We therefore employ the technique of \textit{dynamic feedback linearization} (DFL) to cancel the system's inherent nonlinearities and achieve a linear input-output mapping.

The objective is to design a control law for the input vector $\bs{u}$ defined in Eq. \eqref{eq:input} that forces the position $\bs{P}(t)$ and yaw angle $\psi(t)$ of Ingenuity to track a desired smooth trajectory $(\bs{P}_d(t), \psi_d(t))$. To achieve this, we will first analyze the system's input-output relationship to determine the \textit{relative degree} of each output with respect to the inputs. Then, we will design a \textit{dynamic compensator} with an internal state $\bs{\zeta}$ to augment the system and achieve full relative degree, so as to get a feedback linearizing control law. Finally, we will design an \textit{outer-loop} linear tracking controller to stabilize the tracking error for the linearized system.

\section{Dynamic feedback linearization}
For a general nonlinear system of the form $\dbs{\xi} = \bs{f}(\bs\xi) + \bs{G}(\bs\xi)\bs{u}$ with output $\bs{y} = \bs{h}(\bs\xi)$, the goal of feedback linearization is to find a \textit{coordinate transformation} and a control law that renders the input-output map linear. This is achieved by differentiating each output $y_i$ until at least one input $u_j$ appears. The number of differentiation required is called the \textit{relative degree} $r_i$ of the output $y_i$.

If the total relative degree $r = \sum r_i$ is less than the dimension of the state $n$, the system has residual internal dynamics (\textit{zero dynamics}) that may be unstable. On the other hand, if $r=n$ the system can be fully linearized.

When the control inputs do not appear in a way that allows for direct cancellation of nonlinearities, that is when the \textit{decoupling matrix} is singular, we introduce a \textit{dynamic compensator} with its own internal state $\bs\zeta$ and dynamics $\dbs{\zeta}$. This involves augmenting the system with integrators on the input channels, which leads to an increased dimension of the state space to match the total relative degree.

\subsection{Ingenuity DFL}
We apply the DFL framework to the dynamics of Ingenuity. The outputs we want to control are the position $\bs{P} = (x,y,z)^T$ and yaw angle $\psi$, so we define the output vector as:
\begin{equation}\label{eq:output}
    \bs{y} = \begin{pmatrix}
        x \\ y \\ z \\ \psi
    \end{pmatrix}
\end{equation}
We can now differentiate each component of the output vector $\bs{y}$ until a component of the input vector $\bs{u}$ appears, and determine the corresponding relative degree.

\subsubsection{Yaw channel}
The yaw dynamics are governed by the rotational kinematics and the rotational dynamics. Differentiating the yaw angle $\psi$ once gives:
\[ \dot{y}_4 = \dot{\psi} = \frac{\sin\phi}{\cos\theta}q + \frac{\cos\phi}{\cos\theta}r \]
This expression depends only on the states $\phi, \theta, q, r$ and no inputs appear, so we differentiate again:
\begin{align*}
    \ddot{\psi} = \underbrace{(s_\phi\sec\theta)\dot{q} + (c_\phi\sec\theta)\dot{r}}_{\text{input-dependent}} + \underbrace{(\dot\phi c_\phi\sec\theta + s_\phi\dot\theta\sec\theta\tan\theta)q + (-\dot\phi s_\phi\sec\theta + c_\phi\dot\theta\sec\theta\tan\theta)r}_{\text{state-dependent =} \bs{f}_\psi(\bs\xi)}
\end{align*}
where $\bs{f}(\bs{\xi})$ is the \textit{drift} term, and the input-dependent part can be expressed as:
\[ \begin{pmatrix}
    0 & s_\phi\sec\theta & c_\phi\sec\theta
\end{pmatrix}\dbs{\omega} \]
where $\dbs{\omega}$ is given by the rotational dynamics in Eq. \eqref{eq:euler}. Substituting this expression, we obtain:
\begin{equation}\label{eq:ddot_psi}
    \ddot{\psi} = \bs{f}_\psi(\bs\xi) + \begin{pmatrix}
        0 & s_\phi\sec\theta & c_\phi\sec\theta
    \end{pmatrix}\bs{I}^{-1}\left(\bs\tau_\text{control}^b + \bs\tau_\text{drag}^b - \bs\omega \times \bs{I}\bs\omega\right)
\end{equation}
We see that the control torques $\bs\tau_\text{control}^b$ appear linearly (in particular, the two inputs $u_3=\tau_\theta$ and $u_4=\tau_\psi$), so the relative degree of the yaw channel is $r_\psi = 2$.

\subsubsection{Position channel}
The position dynamics reveal that the system is \textit{underactuated}. From the translational kinematics in Eq. \refeq{eq:kinematics_position}, we have:
\[ \dbs{P} = \bs{R}(\bs\Theta)\bs{V}^b \]
which is independent of the inputs. We therefore differentiate twice:
\begin{align*} 
    \ddbs{P} &= \dbs{R}\bs{V}^b + \bs{R}\dbs{V}^b \\
    &= \cancel{\bs{RS}(\bs\omega)\bs{V}^b} + \bs{R}\left[\frac{1}{m}\left(\bs{F}_\text{thrust}^b + \bs{F}_\text{gravity}^b + \bs{F}_\text{drag}^b \right) - \cancel{\bs\omega \times \bs{V}^b} \right] \\
    &= \frac{1}{m}\bs{R}\begin{pmatrix}
        0 \\ 0 \\ \circledmath{F_T}
    \end{pmatrix} + \begin{pmatrix}
        0 \\ 0 \\ -g
    \end{pmatrix} - \frac{1}{m}\bs{R}\bs{A}_\text{trans}\bs{V}^b
\end{align*}
The input $F_T$ appears, but the control inputs $\tau_\phi, \tau_\theta$ do not. We must continue differentiating:
\[ P^{(3)} = \frac{d}{dt}\left[ \frac{F_T}{m}\bs{R}\bs{e}_3 -g\bs{e}_3 + \frac{1}{m}\bs{R}\bs{F}_\text{drag}^b \right] \]
The time derivatve of the second term is zero. The third term depends on $\bs{V}^b$ and $\bs\omega$, but not on its derivative $\dbs\omega$, and thus will not introduce any new input. As for the first term, we have:
\[ \frac{d}{dt} \left[\frac{F_T}{m}\bs{R}\bs{e}_3\right] = \frac{\circledmath{\dot{F}_T}}{m}\bs{R}\bs{e}_3 + \frac{F_T}{m}\dbs{R}\bs{e}_3 \]
The term $\dot{F}_T$ appears, which is not a control input. We therefore differentiate one last time:
\[ P^{(4)} = \frac{d}{dt}\left[ \frac{\dot{F}_T}{m}\bs{R}\bs{e}_3 + \frac{F_T}{m}\bs{RS}(\bs\omega)\bs{e}_3 + \dots \right] \]
The first term gives:
\[ \frac{d}{dt} \left[ \frac{\dot{F}_T}{m}\bs{R}\bs{e}_3 \right] = \frac{\circledmath{\ddot{F}_T}}{m}\bs{R}\bs{e}_3 + \frac{\dot{F}_T}{m}\dbs{R}\bs{e}_3 \]
Again, we have a non-input term $\ddot{F}_T$. The second term gives:
\[ \frac{d}{dt}\left[ \frac{F_T}{m}\bs{RS}(\bs\omega)\bs{e}_3 \right] = \dots + \frac{F_T}{m}\bs{RS}(\circledmath{\dbs{\omega}})\bs{e}_3 \]
where $\dbs{\omega}$ appears, which is a direct linear function of the input torques $\bs\tau_\text{control}^b$. Therefore, the relative degree of each position channel is $r_x = r_y = r_z = 4$.

\subsubsection{Dynamic compensator}
The appearance of $\ddot{F}_T$ in $\bs{P}^{(4)}$ indicates that the thrust input $F_T$ does not directly influence the outputs in a way that allows for cancellation of nonlinearities. We need to introduce a dynamic compensator for the thrust channel, augmenting the state of the system with two additional controller states:
\begin{equation}\label{eq:compensator_state}
    \bs\zeta = \begin{pmatrix}
        \zeta_1 \\ \zeta_2
    \end{pmatrix} = \begin{pmatrix}
        F_T \\ \dot{F}_T
    \end{pmatrix}
\end{equation}
The dynamics of the compensator are given by a simple integrator chain driven by a new \textit{virtual input} $v_T$:
\begin{equation}\label{eq:compensator_dynamics}
    \dbs{\zeta} = \begin{pmatrix}
        \dot{\zeta}_1 \\ \dot{\zeta}_2
    \end{pmatrix} = \begin{pmatrix}
        \zeta_2 \\ v_T
    \end{pmatrix} = \begin{pmatrix}
        \dot{F}_T \\ \ddot{F}_T
    \end{pmatrix} 
\end{equation}
The thrust $F_T$ applied to Ingenuity is now treated as a state variable $\zeta_1$ that evolves according to the compensator dynamics. The extended state of the system is $(\bs\xi, \bs\zeta)\in\mathbb{R}^{14}$.

\subsubsection{Linearizing control law}
The total relative degree of the augmented system is now:
\[ r = r_x + r_y + r_z + r_\psi = 4 + 4 + 4 + 2 = 14 \]
which equals the dimension of the extended state space ($n_\text{ext} = 12 + 2 = 14$). Therefore, the system is fully state linearizable with no zero dynamics.

We define the extended input vector as:
\begin{equation}\label{eq:extended_input}
    \tilde{\bs{u}} = \begin{pmatrix}
        \ddot{F}_T \\ \tau_\phi \\ \tau_\theta \\ \tau_\psi
    \end{pmatrix}
\end{equation}

The dynamics of the output can now be expressed in the standard affine form:
\begin{equation}
    \begin{pmatrix}
        P^{(4)} \\ \ddot{\psi}
    \end{pmatrix} = \bs{l}(\bs\xi, \bs\zeta) + \bs{J}(\bs\xi, \bs\zeta)\tilde{\bs{u}}
\end{equation}
where $\bs{l}(\bs\xi, \bs\zeta)$ is the drift vector containing all state-dependent terms, and $\bs{J}(\bs\xi,\bs\zeta)$ is the $4 \times 4$ decoupling matrix. 

\paragraph{Decoupling matrix $\bs{J}$} The decoupling matrix $\bs{J}$ contains the coefficients of the extended inputs $\tilde{\bs{u}}$ in the output derivative equations. By carrying out the differentiations in the previous section, we can identify its components:
\begin{equation}
    \bs{J}(\bs{\xi}, \bs{\zeta}) = \begin{bmatrix}\begin{array}{c|c}
        \frac{1}{m}\bs{R}\bs{e}_3 & -\frac{F_T}{m}\bs{R}\bs{S}(\bs{e}_3)\bs{I}^{-1} \\[3pt]
        \cline{1-2} \\[-12pt]
        \bs{0}_{1\times 1} & \begin{pmatrix}
        0 & s_\phi\sec\theta & c_\phi\sec\theta
    \end{pmatrix}\bs{I}^{-1}
    \end{array}\end{bmatrix}
\end{equation}
The block $\bs{J}_{11}$ is a $3 \times 1$ vector representing the effect of $\ddot{F}_T$ on $\bs{P}^{(4)}$. The block $\bs{J}_{12}$ is a $3 \times 3$ matrix representing the effect of the control torques on $\bs{P}^{(4)}$. The block $\bs{J}_{21}$ is zero, since the yaw acceleration is independent of the thrust dynamics $\ddot{F}_T$. The block $\bs{J}_{22}$ is a $1 \times 3$ vector representing the effect of the control torques on $\ddot{\psi}$.

This decoupling matrix is dependent on $F_T$ and becomes singular if $F_T=0$. In this condition, the torques have no effect on the position, and the system is uncontrollable. In practice, we implement a check to disable the controller if the thrust magnitude approaches zero.

\paragraph{Drift vector $\bs{l}$} The drift vector $\bs{l}$ contains all the state-dependent terms from the differentiation that do not multiply the extended inputs. It can be partitioned as:
\[ \bs{l} = \begin{pmatrix}
    \bs{l}_{pos} \\ l_\psi
\end{pmatrix} \]
The yaw drift $l_\psi$ is derived from Eq. \eqref{eq:ddot_psi}:
\begin{equation}
    l_\psi(\bs\xi) = (\dot\phi c_\phi \sec\theta + s_\phi\dot\theta\sec\theta \tan\theta)q + (-\dot\phi s_\phi\sec\theta + c_\phi\dot\theta\sec\theta\tan\theta)r
\end{equation}
The position drift $\bs{l}_\text{pos}$ contains all terms from the fourth derivative of position that are not part of the decoupling matrix. Its full expression is complex, so we highlight only its main components:
\begin{equation}
    \bs{l}_{pos} = \frac{2\dot{F}_T}{m}\bs{R}\bs{S}(\bs\omega)\bs{e}_3 + \frac{F_T}{m}\bs{R}\bs{S}^2(\bs\omega)\bs{e}_3 + \frac{F_T}{m}\bs{R}\bs{S}(\dbs{\omega}_\text{drift})\bs{e}_3 + \dots
\end{equation}
where $\dbs{\omega}_\text{drift} = \bs{I}^{-1}(\bs\tau_\text{drag}^b - \bs\omega \times \bs{I}\bs\omega)$ is the part of the angular acceleration not caused by control inputs. The ellipsis represents higher-order drag terms which are often simplified in implementation.

\paragraph{Control law} The linearizing control law is chosen to cancel these complex terms:
\begin{equation}\label{eq:dfl_law}
    \tilde{\bs{u}} = \bs{J}^{-1}(\bs\xi, \bs\zeta) \left[ \bs{v} - \bs{l}(\bs\xi, \bs\zeta) \right]
\end{equation}
where $\bs{v} = (v_1, v_2, v_3, v_4)^T$ is the new simplified input vector. Applying this control law transforms the input-output dynamics into a set of decoupled linear integrator chains:
\begin{equation}\label{eq:linearized_dynamics}
    \begin{cases}
        x^{(4)} = v_1 \\
        y^{(4)} = v_2 \\
        z^{(4)} = v_3 \\
        \ddot{\psi} = v_4
    \end{cases}
\end{equation}

\subsubsection{Outer-loop tracking controller}
The final step is to design the input $\bs{v}$ to ensure the tracking errors converge to zero. Let the position error be $\bs{e}_P = \bs{P}_d - \bs{P}$ and the yaw error be $e_\psi = \psi_d - \psi$. For the fourth-order position dynamics, we design a controller with feedforward and state feedback on the error dynamics:
\begin{equation}\label{eq:position_controller}
    v_{1,2,3} = \bs{P}_d^{(4)} + \bs{K}_3 \bs{e}_P^{(3)} + \bs{K}_2 \ddbs{e}_P + \bs{K}_1 \dbs{e}_P + \bs{K}_0 \bs{e}_P
\end{equation}
where $v_{1,2,3} = (v_1, v_2, v_3)^T$ and $\bs{K}_i = \text{diag}(k_{i,x}, k_{i,y}, k_{i,z})$ are diagonal gain matrices.

For the second-order yaw dynamics, we use a PD controller with feedforward:
\begin{equation}\label{eq:yaw_controller}
    v_4 = \ddot{\psi}_d + k_{d, \psi}\dot{e}_\psi + k_{p, \psi}e_\psi
\end{equation}

The gains in Eqs. \eqref{eq:position_controller} and \eqref{eq:yaw_controller} are chosen to place the poles of the respective error characteristic equations in the left-half of the complex plane, ensuring exponential convergence of the tracking errors to zero.

\section{Control architecture summary}
The complete control architecture is summarized in Figure \ref{fig:control_diagram}. The outer loop computes the virtual input $\bs{v}$ based on the tracking error. The inner loop computes the state-dependent terms $\bs{l}$ and $\bs{J}$ and uses the DFL law \eqref{eq:dfl_law} to calculate the extended inputs $\tilde{\bs{u}}$. The torque commands are sent to the helicopter, while the thrust rate command $v_T$ is integrated through the dynamic compensator to produce the thrust input $F_T$.

*TODO: add block scheme*
%\begin{figure}[htbp]
%\centering
%\includegraphics[width=0.9\textwidth]{images/dfl_diagram_placeholder.png}
%\caption{Block diagram of the complete DFL control system.}
%\label{fig:control_diagram}
%\end{figure}